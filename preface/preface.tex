%% The first command in your LaTeX source must be the \documentclass command.
%%
%% Options:
%% twocolumn : Two column layout.
%% hf: enable header and footer.
\documentclass[
% twocolumn,
% hf,
]{ceurart}

%%
%% One can fix some overfulls
\sloppy

%%
%% Minted listings support 
%% Need pygment <http://pygments.org/> <http://pypi.python.org/pypi/Pygments>
\usepackage{listings}
%% auto break lines
\lstset{breaklines=true}

%%
%% end of the preamble, start of the body of the document source.
\begin{document}

%%
%% Rights management information.
%% CC-BY is default license.
\copyrightyear{2025}
\copyrightclause{Copyright for this paper by its authors.
  Use permitted under Creative Commons License Attribution 4.0
  International (CC BY 4.0).}

%%
%% This command is for the conference information
\conference{41th International Conference on Logic Programming, September 9-13, 2025, Rende, Italy}

%% can use this document as the template for p
%% The "title" command
\title{Preface of the Joint Proceedings of the Workshops and Doctoral Consortium of the 41th International Conference on Logic Programming}


%%
%% The "author" command and its associated commands are used to define
%% the authors and their affiliations.
\author[1]{Pierangela Bruno}[%
orcid=0000-0002-0832-0151,
email=pierangela.bruno@unical.it,
]
\address[2]{University of Calabria, Italy}


\author[2]{Jorge Fandinno}[%
orcid=0000-0002-3917-8717,
email=jfandinno@unomaha.edu,
]
\address[2]{University of Nebraska Omaha, USA}

\def\changemargin#1#2{\list{}{\rightmargin#2\leftmargin#1}\item[]}
\let\endchangemargin=\endlist 

\newcommand{\workshopp}[2]{\section{#1}\vspace{-8pt}\begin{changemargin}{19pt}{8pt} 
\rm\normalsize\it #2\end{changemargin}}
\newcommand{\workshop}[3]{\workshopp{#2}{#1 #2 (#3)}}


%%
%% The abstract is a short summary of the work to be presented in the
%% article.
\begin{abstract}
This volume collects the papers accepted for publication at XXX workshops and the Doctoral Consortium associated with the 41th International Conference on Logic Programming (ICLP 2025).
%
The events took place in Rende, Italy, on September 9-13, 2025.
%
This volume contains a total of 40 papers, 19 of which were presented at the workshops and 21 at the Doctoral Consortium.
%
Of the 19 workshop papers, 8 were long papers, another 8 were short papers, and 3 were abstracts corresponding to invited talks.
%
The Doctoral Consortium papers were all short papers.
\end{abstract}

\maketitle

\workshop{32nd Workshop on}{Experimental Evaluation of Algorithms for Solving Problems with Combinatorial Explosion}{RCRA 2025}

\noindent
Many problems in Artificial Intelligence show an exponential explosion of the search space, and are addressed with algorithms that aim at an effective exploration.  Research in Artificial Intelligence has focused on experimental evaluation of algorithms, and the implementation of systems for solving such problems.

 

Scope of the 32nd RCRA workshop on Experimental Evaluation of Algorithms for Solving Problems with Combinatorial Explosion (RCRA 2025) is fostering the cross-fertilisation of ideas from different areas, analyzing and comparing models and algorithms from an experimental viewpoint.

 

Out of 7 papers presented at RCRA 2025, this volume contains 1 regular paper and 5 short papers.

\subsection*{Chairs}
\begin{itemize}
  \item Marco Maratea, \emph{University of Calabria}, Italy
  \item Luciano Serafini, \emph{FBK}, Italy
  \item Mauro Vallati, \emph{University of Huddersfield}, UK
\end{itemize}


\subsection*{Program Committee}
\begin{itemize}
  \item Johannes P. Wallner, TU Graz
  \item Stefania Costantini, \emph{Università degli Studi dell'Aquila}
  \item Valentino Santucci, \emph{University for Foreigners of Perugia}
  \item Francesco Calimeri, \emph{University of Calabria}
  \item Giuseppe Mazzotta, \emph{University of Calabria}
  \item Ivan Serina, \emph{University of Brescia}
  \item Francesco Percassi, \emph{University of Huddersfield}
  \item Francesco Santini, \emph{University of Perugia}
  \item Wolfgang Faber, \emph{Alpen-Adria-Universität Klagenfurt}
  \item Alice Tarzariol, \emph{Alpen-Adria-Universität Klagenfurt}
  \item Andrea Formisano, \emph{Università di Udine}
  \item Alessandro Bertagnon, \emph{University of Ferrara}
  \item Carmine Dodaro, \emph{University of Calabria}
  \item Luigi Bonassi, \emph{University of Oxford}
  \item Leonardo Lamanna, \emph{Fondazione Bruno Kessler}
  \item Matteo Cardellini, \emph{Università degli Studi di Genova}
\end{itemize}


\workshop{18th Workshop on}{Answer Set Programming and Other Computing Paradigms}{ASPOCP 2025}

\noindent
The Workshop on Answer Set Programming and Other Computing Paradigms (ASPOCP) has been running for almost twenty years, providing a well-established forum with a strong program committee and active community. ASPOCP covers ASP and its connections to other paradigms such as SAT, SMT, QBF, FO(ID), and constraint programming, as well as applications and extensions.

The eighteenth edition of ASPOCP comprises a total of six works, consisting of four long papers, one extended abstract from an already published paper and an extended abstract corresponding to the invited talk.

\subsection*{Chairs}
\begin{itemize}
 \item Brais Muñiz, \emph{University of Coru\~na}, Spain
 \item Alice Tarzariol, \emph{University of Klagenfurt}, Austria
\end{itemize}


\subsection*{Program Committee}
\begin{itemize}
  \item Mario Alviano, \emph{University of Calabria}
  \item Pedro Cabalar, \emph{University of Coru\~na}
  \item Tran Cao, \emph{New Mexico State University}
  \item Francesco Cauteruccio, \emph{University of Salerno}
  \item Stefania Costantini, \emph{University of L'Aquila}
  \item Carmine Dodaro, \emph{University of Calabria}
  \item Esra Erdem, \emph{Sabanci University}
  \item Cristina Feier, \emph{Technical University of Cluj-Napoca}
  \item Johannes K. Fichte, \emph{Linköping University}
  \item Tobias Geibinger, \emph{Vienna University of Technology}
  \item Giovambattista Ianni, \emph{University of Calabria}
  \item Tomi Janhunen, \emph{Tampere University}
  \item Vladimir Lifschitz, \emph{University of Texas}
  \item Marco Maratea, \emph{University of Calabria}
  \item Michael Morak, \emph{University of Klagenfurt}
  \item Orkunt Sabuncu, \emph{Potassco Solutions Turkey}
  \item Konstantin Schekotihin, \emph{University of Klagenfurt}
  \item Van-Giang Trinh, \emph{Inria Saclay}
  \item Jia-Huai You, \emph{University of Alberta}
  \item Johannes P. Wallner, \emph{TU Graz}
\end{itemize}

\workshop{12th Workshop on}{Probabilistic Logic Programming}{PLP 2025}

\subsection*{Chairs}
\begin{itemize}
  \item Damiano Azzolini, \emph{University of Ferrara}, Italy
  \item Markus Hecher, \emph{CNRS, Artois University}, France
\end{itemize}


\subsection*{Program Committee}
\begin{itemize}
  \item Elena Bellodi, \emph{University of Ferrara}, Italy
  \item Alice Bizzarri, \emph{University of Ferrara}, Italy
  \item Fabio Gagliardi Cozman, \emph{University of São Paulo}, Brasil
  \item James Cussens, \emph{University of Bristol}, England
  \item Antonio Ielo, \emph{University of Calabria}, Italy
  \item Rafael Kiesel, \emph{TU Wien}, Austria
  \item Denis Maua, \emph{University of São Paulo}, Brasil
  \item Fabrizio Riguzzi, \emph{University of Ferrara}, Italy
  \item Kilian Rückschloß, \emph{University of Tübingen}, Germany
  \item Theresa Swift, \emph{Coherent Knowledge, Inc.}, USA
  \item Joost Vennekens, \emph{KU Leuven}, Belgium
  \item Felix Weitkämper, \emph{University of München}, Germany
  \item Riccardo Zese, \emph{University of Ferrara}, Italy
\end{itemize}

% \workshop{9th Workshop on}{Advances in Argumentation in Artificial Intelligence}{AI$^3$ 2025}

% \subsection*{Program Chairs}
% \begin{itemize}
%   \item Mario Alviano, \emph{University of Calabria}, Italy
%   \item Bettina Fazzinga, \emph{University of Calabria}, Italy
% \end{itemize}

% \subsection*{Publicity Chairs}
% \begin{itemize}
%   \item Manuel Alejandro Borroto Santana, \emph{University of Calabria}, Italy
% \end{itemize}


% \subsection*{Program Committee}
% \begin{itemize}
%   \item Gianvincenzo Alfano, \emph{University of Calabria}
%   \item Pietro Baroni, \emph{University of Brescia}
%   \item Stefano Bistarelli, \emph{University of Perugia}
%   \item Sergio Flesca, \emph{DIMES - UNICAL}
%   \item Massimiliano Giacomin, \emph{University of Brescia}
%   \item Daniele Porello, \emph{University of Genoa}
%   \item Carlo Proietti, \emph{CNR, ILC}
%   \item Francesco Santini, \emph{University of Perugia}
%   \item Carlo Taticchi, \emph{University of Perugia}
%   \item Alice Toniolo, \emph{University of St Andrews}
%   \item Paolo Torroni, \emph{University of Bologna}
%   \item Mauro Vallati, \emph{University of Huddersfield}
% \end{itemize}

\workshopp{Prolog Education}{3rd Prolog Education Workshop (PEG 2025)}

\noindent
This part of the volume contains the papers presented at the Third Prolog Education Workshop, PEG 2025, one of the initiatives of the Prolog Education Group 2.0 (PEG 2.0). We received 9 submissions out of which 6 papers were accepted as regular papers, and 2 as short papers. In addition to the technical papers, the workshop included two invited talks: Verónica Dahl (Simon Fraser University, Canada) delivered "PEG 2.0: Future-gazing through a socio-linguistic and historical lens" and Theresa Swift (Universidade Nova de Lisboa, Portugal) presented "LLM-Assisted Education for a Low-Resource Logic Programming Language".


\subsection*{Chairs}
\begin{itemize}
  \item Laura A. Cecchi, \emph{Universidad Nacional del Comahue}, Argentina
  \item José F. Morales, \emph{T.U. Madrid (UPM) and IMDEA Software Institute}, Spain
\end{itemize}

\subsection*{Program Committee}
\begin{itemize}
\item Salvador Abreu, \emph{Universidade de Evora}
\item Joaquín Arias, \emph{CETINIA, Universidad Rey Juan Carlos}
\item Asya Astanova, \emph{Plovdiv University}
\item Roberta Calegari, \emph{Università di Bologna}
\item Stefania Costantini, \emph{Università degli Studi dell'Aquila}
\item Verónica Dahl, \emph{Simon Fraser University}
\item Jacinto Dávila, \emph{Universidad de Los Andes}
\item Włodek Drabent, \emph{Institute of Computer Science, Polish Academy of Sciences}
\item Atanas Dukovski, \emph{Bulgarian Academy of Sciences Institute of Information and Communication}
\item Michael Genesereth, \emph{Stanford University}
\item Gopal Gupta, \emph{University of Texas at Dallas}
\item Angelo Ferrando, \emph{Università degli Studi di Modena e Reggio Emilia}
\item Jason Hemann, \emph{Seton Hall University}
\item Manuel Hermenegildo, \emph{T.U. Madrid (UPM) and IMDEA Software Institute}
\item Bharat Jayaraman, \emph{Amrita Institute of Advanced Research}
\item Christian Jendreiko, \emph{HSD University of Applied Sciences}
\item Bob Kowalski, \emph{Imperial College London}
\item Pedro Lopez, \emph{T.U. Madrid (UPM) and IMDEA Software Institute}
\item Fernando Sáenz-Perez, \emph{Universidad Complutense de Madrid}
\item Theresa Swift, \emph{Universidade Nova de Lisboa}
\item Veneta Tabakova-Komsalova, \emph{Plovdiv University}
\item Paul Tarau, \emph{University of North Texas}
\item David S. Warren, \emph{Stony Brook University}
\item Felix Weitkämper, \emph{Universität München}
\item Jan Wielemaker, \emph{Vrije Universiteit Amsterdam}
\item Adam Wyner, \emph{Swansea University}
\end{itemize}

% \workshop{2nd Workshop on}{Prolog Improvement Proposals}{PIPs 2025}

\workshop{1st Workshop on}{Logic Programming and Legal Reasoning}{LPLR 2025}

\noindent
This workshop explores the representation of legal rules and the automation of reasoning over them through logic programming. Laws and regulations are complex, large-scale, and central to most human activities, making computational support essential for tasks such as compliance checking, decision support, and normative reasoning. By combining perspectives from law and computer science, the event provides a forum to discuss advances in theory and applications. The program featured two long papers and one short paper, highlighting recent research results and fostering dialogue on innovative approaches in this interdisciplinary and rapidly evolving area.

\subsection*{Chairs}
\begin{itemize}
\item Ilias Tachmazidis, \emph{University of Huddersfield}, UK

\item Sotiris Batsakis, \emph{University of Huddersfield}, UK

\item Livio Robaldo, \emph{University of Swansea}, UK

\item Emmanuel Papadakis, \emph{University of Huddersfield}, UK

\item Adam Wyner, \emph{University of Swansea}, UK
\end{itemize}

\workshop{1st Workshop on}{Cognitive Architectures for Robotics: LLMs and Logic in Action}{CARLA 2025}

\subsection*{Chairs}
\begin{itemize}
\item Fabrizio Lo Scudo, \emph{University of Calabria}, Italy

\item Sotirios Batsakis, \emph{Hellenic Mediterranean University}, Greece

\item Manuel Alejandro Borroto Santana, \emph{University of Calabria}, Italy
\end{itemize}

\workshopp{Doctoral Consortium}{21st Doctoral Consortium on Logic Programming}

\subsection*{Chairs}
\begin{itemize}
\item Alice Tarzariol, \emph{University of Klagenfurt}, Austria

\item Markus Hecher, \emph{University of Artois, CNRS, Computer Science Research Center of Lens (CRIL)}, France
\end{itemize}

\subsection*{Program Committee}
\begin{itemize}
\item Mario Alviano, \emph{University of Calabria}
\item Damiano Azzolini, \emph{University of Ferrara}
\item Marina De Vos, \emph{University of Bath}
\item Carmine Dodaro, \emph{University of Calabria}
\item Agostino Dovier, \emph{University of Udine}
\item Wolfgang Faber, \emph{University of Klagenfurt}
\item Francesco Fabiano, \emph{New Mexico State University}
\item Cristina Feier, \emph{Technical University of Cluj-Napoca}
\item Johannes Fichte, \emph{Linköping University}
\item Sarah Alice Gaggl, \emph{TU Dresden}
\item Luca Geatti, \emph{University of Udine}
\item Tobias Geibinger, \emph{Vienna University of Technology}
\item Laura Giordano, \emph{Università del Piemonte Orientale}
\item Eleonora Iotti, \emph{University of Parma}
\item Vladimir Lifschitz, \emph{University of Texas at Austin}
\item Yanhong Liu, \emph{Stony Brook University}
\item Marco Maratea, \emph{University of Calabria}
\item Michael Morak, \emph{University of Klagenfurt}
\item Jose Morales, \emph{IMDEA Software Research Institute}
\item Zeynep Saribatur, \emph{TU Wien}
\item Torsten Schaub, \emph{University of Potsdam}
\item Frank Valencia, \emph{LIX, Ecole Polytechnique}
\item Stefan Woltran, \emph{TU Wien}
\end{itemize}

\end{document}

%%
%% End of file
